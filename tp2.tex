\documentclass[a4paper,10pt]{article}
\input{nahuelMacros.tex}
\usepackage{graphicx}
\usepackage[dvipsnames]{xcolor}
\begin{document}

\paragraph*{Como pensamos representar los elementos:}

\begin{itemize}
    \item $C$: Conjunto de las carreras de grado.


          Lo representamos como Trie que se accede desde la instancia siu.
    \item $c$: Nombre de carrera.


          Lo representamos como string. Por lo tanto |c| indica el largo del nombre de la carrera.
    \item $M_c$: Conjunto de las materias del grado c

          Lo representamos como Trie que se accede desde la instancia de Carrera (Debajo desarrollaremos un ejemplo concreto)
    \item $N_m$: Conjunto de nombres de la materia.

          Los caracteres de estos nombres seran los nodos del Trie anterior. Los nodos significativos apuntaran a las materias respectivas.
    \item $n$: Nombre de la materia

          Lo representamos como un string, por lo tanto |n| indica el largo del nombre de la materia.
    \item $E$ y $E_m$: los representaremos como enterous.
\end{itemize}
\salto{\baselineskip}
Libretas universitarias $\rightarrow$ Trie acotado, lo que implica que las operaciones del trie son de O(1)

{\small El nodo significativo de cada libreta apuntara a la instancia de Estudiante}

\salto{\baselineskip}

NombreCarreras $\rightarrow$ Trie no acotado, operaciones O(log(n))

{\small El nodo significativo apuntara a la instancia de la Carrera}

\salto{\baselineskip}

NombreMaterias $\rightarrow$ Trie no acotado, operaciones O(log(n))

{\small El nodo significativo apuntara a la instancia de Materia}

\salto{\baselineskip}

Veamos el siguiente ejemplo:
\salto{\baselineskip}

Tenemos la carrera fisica y la carrera matematica, en la primera tenemos la materia

"Matematica 1", y en la segunda tenemos 'Analisis 1', ambas siendo la misma materia.

A traves del trie NombreCarreras accedemos a una instancia de la clase Carrera

en el nodo significativo. Esta clase nos permite acceder a su trie NombreMaterias, y en el nodo significativo acceder a la instancia de la materia.



\pagebreak
\vspace*{15ex}
En este caso particular la materia "Matematica 1 / Analisis 1"

es accedida desde dos caminos distintos:
\salto{\baselineskip}
\begin{figure*}[h]
    \centering
    \includegraphics[width=0.5\textwidth]{diagrama1.png}
\end{figure*}
\pagebreak
\section*{Dudas:}
\begin{figure*}[h]
    \centering
    \includegraphics[width=0.7\textwidth]{duda1.png}
\end{figure*}
\pagebreak
\section*{Pseudoresoluciones:}
{\vspace*{-2ex}\hspace*{4em} \small Las que pase a limpio al menos.}


\subsection*{7. carreras(in sistema: SistemaSIU):seq\smm{string}}
\{

\{

    ArrayList lista = new ArrayList() \hfill $\color{Purple}\longleftarrow O(1)$
    
    Iterador it= nuevo iterador del sistemaSIU \hfill $\color{Purple}\longleftarrow O(1)$

    while(iterador.haySiguiente()){\hfill $\color{Purple}\longleftarrow O(\sum_{c\in C}^{} |c|)$

        \hspace*{1.5em} lista.add(iterador.siguiente())

    }

    return lista \hfill $\color{Purple}\longleftarrow O(1)$

\}\hfill $\color{Purple} O(1+1+\sum_{c\in C}^{} |c| +1 )\equiv O(\sum_{c\in C}^{} |c|)$

\noindent\}

\salto{\baselineskip}
\anotacionns[ForestGreen]{Nota: el iterador del trie devuelve los strings ordenador de forma lexicografica.}

\subsection*{8. materias(in sistema: SistemaSIU, in carrera: string):seq\smm{string}}
\{

\{

    Materias materia = Carreras.buscar(carrera) \hfill $\color{Purple}\longleftarrow O(|carrera|)$

    ArrayList lista = new ArrayList() \hfill $\color{Purple}\longleftarrow O(1)$
    
    Iterador it= nuevo iterador del la carrera \hfill $\color{Purple}\longleftarrow O(1)$

    while(iterador.haySiguiente()){\hfill $\color{Purple}\longleftarrow O(\sum_{m_c\in M_c}^{} |m_c|)$

        \hspace*{1.5em} lista.add(iterador.siguiente())

    }

    return lista \hfill $\color{Purple}\longleftarrow O(1)$

\}

\noindent\}\hfill $\color{Purple} O(|carrera|+1+1+\sum_{m_c\in M_c}^{} |m_c| +1 )\equiv O(|carrera|+\sum_{m_c\in M_c}^{} |m_c|)$

\salto{\baselineskip}
\anotacionns[ForestGreen]{Nota: el iterador del trie devuelve los strings ordenador de forma lexicografica.}
\end{document}